\documentclass[12pt,a4paper]{article}

% ==================== PACKAGES ====================
\usepackage[utf8]{inputenc}
\usepackage[T1]{fontenc}
\usepackage[french]{babel}
\usepackage{geometry}
\usepackage{graphicx}
\usepackage{hyperref}
\usepackage{listings}
\usepackage{xcolor}
\usepackage{tikz}
\usepackage{tcolorbox}
\usepackage{enumitem}
\usepackage{booktabs}
\usepackage{multirow}
\usepackage{fancyhdr}
\usepackage{titlesec}
\usepackage{amsmath}
\usepackage{amssymb}

% ==================== CONFIGURATION ====================
\geometry{margin=2cm}
\hypersetup{colorlinks=true, linkcolor=blue, urlcolor=cyan}

% Couleurs personnalisées
\definecolor{codegreen}{rgb}{0,0.6,0}
\definecolor{codegray}{rgb}{0.5,0.5,0.5}
\definecolor{codepurple}{rgb}{0.58,0,0.82}
\definecolor{backcolour}{rgb}{0.95,0.95,0.92}
\definecolor{aquablue}{RGB}{0,119,182}
\definecolor{alertred}{RGB}{230,57,70}
\definecolor{successgreen}{RGB}{40,167,69}

% Configuration des listings
\lstdefinestyle{mystyle}{
    backgroundcolor=\color{backcolour},
    commentstyle=\color{codegreen},
    keywordstyle=\color{codepurple},
    numberstyle=\tiny\color{codegray},
    stringstyle=\color{codegreen},
    basicstyle=\ttfamily\footnotesize,
    breakatwhitespace=false,
    breaklines=true,
    captionpos=b,
    keepspaces=true,
    numbers=left,
    numbersep=5pt,
    showspaces=false,
    showstringspaces=false,
    showtabs=false,
    tabsize=2,
    frame=single
}
\lstset{style=mystyle}

% Configuration des boîtes
\tcbuselibrary{skins,breakable}

% En-têtes et pieds de page
\pagestyle{fancy}
\fancyhf{}
\fancyhead[L]{\textbf{AquaWatch-MS}}
\fancyhead[R]{Résumé du Projet}
\fancyfoot[C]{\thepage}

% ==================== DOCUMENT ====================
\begin{document}

% ==================== PAGE DE TITRE ====================
\begin{titlepage}
    \centering
    \vspace*{2cm}
    
    {\Huge\textbf{\textcolor{aquablue}{AquaWatch-MS}}}\\[0.5cm]
    {\Large\textit{Plateforme Modulaire de Surveillance et Prédiction\\de la Qualité de l'Eau}}
    
    \vspace{1.5cm}
    
    \begin{tcolorbox}[colback=aquablue!10, colframe=aquablue, width=0.9\textwidth]
        \centering
        \textbf{Architecture Microservices | Machine Learning | IoT | SIG}
    \end{tcolorbox}
    
    \vspace{2cm}
    
    \begin{tabular}{ll}
        \textbf{Équipe:} & \\
        $\bullet$ Hamza & STModel + Infrastructure \\
        $\bullet$ Bilal & Capteurs + Satellite \\
        $\bullet$ Yassin & Alertes + API-SIG \\
    \end{tabular}
    
    \vspace{2cm}
    
    {\large EMSI - 5ème Année}\\
    {\large Module: ML + DM + Microservices}\\[0.5cm]
    {\large Décembre 2025}
    
    \vfill
\end{titlepage}

% ==================== TABLE DES MATIÈRES ====================
\tableofcontents
\newpage

% ==================== SECTION 1: PRÉSENTATION GÉNÉRALE ====================
\section{Présentation Générale du Projet}

\subsection{Objectif Principal}
\begin{tcolorbox}[colback=successgreen!10, colframe=successgreen, title=\textbf{Mission d'AquaWatch}]
Développer une \textbf{plateforme automatisée} pour:
\begin{itemize}[noitemsep]
    \item \textbf{Surveiller} la qualité de l'eau en temps réel via capteurs IoT et données satellites
    \item \textbf{Prédire} les variations à court terme avec des modèles spatio-temporels (LSTM)
    \item \textbf{Alerter} automatiquement en cas de dépassement des seuils OMS
    \item \textbf{Visualiser} l'état des masses d'eau sur interface cartographique
\end{itemize}
\end{tcolorbox}

\subsection{Contexte et Problématique}
La surveillance de la qualité de l'eau est cruciale pour:
\begin{itemize}
    \item La santé publique (normes OMS)
    \item La protection environnementale
    \item La gestion des ressources hydriques
\end{itemize}

\textbf{Problématique:} Comment automatiser la surveillance et anticiper les problèmes de qualité d'eau en combinant:
\begin{enumerate}
    \item Données IoT temps réel (capteurs)
    \item Images satellites (Sentinel-2, Copernicus)
    \item Intelligence artificielle (prédiction)
\end{enumerate}

\subsection{Zones Pilotes au Maroc}
\begin{center}
\begin{tabular}{|l|l|l|}
\hline
\textbf{Zone} & \textbf{Type} & \textbf{Coordonnées} \\
\hline
Casablanca & Urbaine côtière & 33.5731, -7.5898 \\
Rabat & Capitale & 34.0209, -6.8416 \\
Agadir & Côtière & 30.4278, -9.5981 \\
Tanger & Port & 35.7595, -5.8340 \\
\hline
\end{tabular}
\end{center}

% ==================== SECTION 2: ARCHITECTURE ====================
\section{Architecture Microservices}

\subsection{Vue d'Ensemble}

\begin{tcolorbox}[colback=blue!5, colframe=aquablue, title=\textbf{Architecture 5 Services}]
\begin{center}
\begin{tikzpicture}[node distance=2cm, auto, thick]
    % Nodes
    \node[draw, rectangle, fill=orange!30, minimum width=2.5cm, minimum height=1cm] (capteurs) {Capteurs\\Port 8001};
    \node[draw, rectangle, fill=cyan!30, minimum width=2.5cm, minimum height=1cm, right of=capteurs, xshift=2cm] (satellite) {Satellite\\Port 8002};
    \node[draw, rectangle, fill=purple!30, minimum width=2.5cm, minimum height=1cm, below of=capteurs, xshift=2cm] (stmodel) {STModel\\Port 8003};
    \node[draw, rectangle, fill=red!30, minimum width=2.5cm, minimum height=1cm, below of=stmodel, xshift=-2cm] (alertes) {Alertes\\Port 8004};
    \node[draw, rectangle, fill=green!30, minimum width=2.5cm, minimum height=1cm, below of=stmodel, xshift=2cm] (sig) {API-SIG\\Port 8005};
    
    % Arrows
    \draw[->] (capteurs) -- (stmodel);
    \draw[->] (satellite) -- (stmodel);
    \draw[->] (stmodel) -- (alertes);
    \draw[->] (stmodel) -- (sig);
\end{tikzpicture}
\end{center}
\end{tcolorbox}

\subsection{Description des 5 Microservices}

\subsubsection{Service Capteurs (Port 8001) - Bilal}
\begin{itemize}
    \item \textbf{Rôle:} Collecte des données IoT temps réel
    \item \textbf{Technologies:} Node.js, Express.js, TimescaleDB
    \item \textbf{Paramètres mesurés:} pH, température, turbidité, oxygène dissous
    \item \textbf{Communication:} MQTT (simulation) + API REST
    \item \textbf{Base de données:} TimescaleDB (hypertables pour séries temporelles)
\end{itemize}

\subsubsection{Service Satellite (Port 8002) - Bilal}
\begin{itemize}
    \item \textbf{Rôle:} Traitement images satellites Sentinel-2
    \item \textbf{Technologies:} Python, FastAPI, MongoDB, MinIO
    \item \textbf{Indices calculés:}
    \begin{itemize}
        \item NDWI (Normalized Difference Water Index): Détection eau
        \item Chlorophylle: Mesure algues
        \item Turbidité satellite: Clarté de l'eau
    \end{itemize}
    \item \textbf{Stockage:} MongoDB (métadonnées) + MinIO (images)
\end{itemize}

\subsubsection{Service STModel (Port 8003) - Hamza}
\begin{itemize}
    \item \textbf{Rôle:} Prédiction qualité eau via Machine Learning
    \item \textbf{Technologies:} Python, FastAPI, PyTorch, PostgreSQL
    \item \textbf{Modèle:} LSTM (Long Short-Term Memory)
    \item \textbf{Input:} 14 jours d'historique (capteurs + satellites)
    \item \textbf{Output:} Score de qualité prédit (jour 15)
\end{itemize}

\subsubsection{Service Alertes (Port 8004) - Yassin}
\begin{itemize}
    \item \textbf{Rôle:} Gestion et envoi d'alertes automatiques
    \item \textbf{Technologies:} Node.js, Express.js, PostgreSQL, Redis
    \item \textbf{Types d'alertes:}
    \begin{itemize}
        \item WATER\_QUALITY\_BAD: Qualité dégradée
        \item PH\_CRITICAL: pH dangereux (<6 ou >9)
        \item POLLUTION\_RISK: Risque détecté par ML
        \item SENSOR\_OFFLINE: Capteur hors ligne
    \end{itemize}
    \item \textbf{Canaux:} Email, SMS, Webhooks
\end{itemize}

\subsubsection{Service API-SIG (Port 8005) - Yassin}
\begin{itemize}
    \item \textbf{Rôle:} Interface cartographique interactive
    \item \textbf{Technologies:} Next.js, PostGIS, Leaflet.js
    \item \textbf{Fonctionnalités:}
    \begin{itemize}
        \item Carte interactive des zones
        \item Visualisation qualité par couleur
        \item API GeoJSON
    \end{itemize}
\end{itemize}

\subsection{Infrastructure et Bases de Données}

\begin{center}
\begin{tabular}{|l|l|l|l|}
\hline
\textbf{Service} & \textbf{Base de Données} & \textbf{Port} & \textbf{Type} \\
\hline
Capteurs & TimescaleDB & 5433 & Séries temporelles \\
Satellite & MongoDB & 27017 & Documents \\
STModel & PostgreSQL & 5434 & Relationnel \\
Alertes & PostgreSQL & 5435 & Relationnel \\
API-SIG & PostGIS & 5436 & Géospatial \\
\hline
\multicolumn{4}{|l|}{\textbf{Services communs:}} \\
\hline
Redis & Cache & 6379 & File de messages \\
MinIO & Stockage objets & 9000/9001 & S3-compatible \\
GeoServer & Cartographie & 8080 & WMS/WFS \\
\hline
\end{tabular}
\end{center}

% ==================== SECTION 3: TECHNOLOGIES ====================
\section{Stack Technologique}

\subsection{Backend}
\begin{center}
\begin{tabular}{|l|l|l|}
\hline
\textbf{Technologie} & \textbf{Usage} & \textbf{Service} \\
\hline
Node.js + Express.js & API REST & Capteurs, Alertes \\
Python + FastAPI & API REST + ML & Satellite, STModel \\
Sequelize & ORM Node.js & Capteurs, Alertes \\
SQLAlchemy & ORM Python & STModel \\
\hline
\end{tabular}
\end{center}

\subsection{Machine Learning}
\begin{center}
\begin{tabular}{|l|l|}
\hline
\textbf{Technologie} & \textbf{Usage} \\
\hline
PyTorch 2.4.1 & Framework Deep Learning \\
PyTorch Lightning & Entraînement simplifié \\
Scikit-learn & Prétraitement, métriques \\
NumPy & Calcul numérique \\
Pandas & Manipulation données \\
\hline
\end{tabular}
\end{center}

\subsection{Bases de Données}
\begin{itemize}
    \item \textbf{TimescaleDB:} Extension PostgreSQL optimisée pour séries temporelles (hypertables)
    \item \textbf{MongoDB:} Base NoSQL pour documents JSON (images satellites)
    \item \textbf{PostgreSQL:} Base relationnelle classique
    \item \textbf{PostGIS:} Extension PostgreSQL pour données géospatiales
    \item \textbf{Redis:} Cache et file de messages entre services
\end{itemize}

\subsection{DevOps et Infrastructure}
\begin{itemize}
    \item \textbf{Docker + Docker Compose:} Conteneurisation des services
    \item \textbf{MinIO:} Stockage compatible S3 pour images satellites
    \item \textbf{GeoServer:} Serveur de données cartographiques
    \item \textbf{Git + GitFlow:} Gestion de versions
\end{itemize}

\subsection{Frontend}
\begin{itemize}
    \item \textbf{Next.js:} Framework React pour API-SIG
    \item \textbf{Leaflet.js:} Cartes interactives
    \item \textbf{Chart.js:} Graphiques de données
    \item \textbf{WebSockets:} Temps réel
\end{itemize}

% ==================== SECTION 4: MODÈLE ML ====================
\section{Modèle de Machine Learning (STModel)}

\subsection{Architecture du Modèle LSTM}

\begin{tcolorbox}[colback=purple!5, colframe=purple!70, title=\textbf{Architecture WaterQualityLSTM}]
\begin{lstlisting}[language=Python]
class WaterQualityLSTM(nn.Module):
    def __init__(self, input_size=11, hidden_size=64, num_layers=2):
        super(WaterQualityLSTM, self).__init__()
        
        # LSTM 2 couches avec dropout
        self.lstm = nn.LSTM(
            input_size=input_size,    # 11 features
            hidden_size=hidden_size,  # 64 neurones
            num_layers=num_layers,    # 2 couches
            batch_first=True,
            dropout=0.2
        )
        
        # Couche finale
        self.fc = nn.Linear(hidden_size, 1)
\end{lstlisting}
\end{tcolorbox}

\subsection{Features d'Entrée (11 paramètres)}

\begin{center}
\begin{tabular}{|l|l|l|}
\hline
\textbf{Type} & \textbf{Feature} & \textbf{Description} \\
\hline
\multirow{8}{*}{Capteurs (8)} & pH & Acidité de l'eau \\
& oxygene\_dissous & O2 dans l'eau \\
& COD & Demande chimique en oxygène \\
& CODMn & COD au permanganate \\
& NH4N & Ammonium \\
& TPH & Phosphore total \\
& DIP & Phosphore inorganique dissous \\
& DIN & Azote inorganique dissous \\
\hline
\multirow{3}{*}{Satellites (3)} & NDWI & Index différentiel normalisé eau \\
& chlorophyll\_index & Indice chlorophylle \\
& turbidity\_index & Indice turbidité \\
\hline
\end{tabular}
\end{center}

\subsection{Pipeline de Prétraitement}

\begin{enumerate}
    \item \textbf{Nettoyage:} Suppression valeurs manquantes/aberrantes
    \item \textbf{Fusion:} Combinaison données capteurs + satellites
    \item \textbf{Interpolation temporelle:} Alignement des dates
    \item \textbf{Normalisation:} MinMaxScaler (valeurs entre 0 et 1)
    \item \textbf{Création séquences:} Fenêtre glissante de 14 jours
\end{enumerate}

\subsection{Configuration d'Entraînement}

\begin{center}
\begin{tabular}{|l|l|}
\hline
\textbf{Paramètre} & \textbf{Valeur} \\
\hline
Fonction de perte & MSE (Mean Squared Error) \\
Optimiseur & Adam \\
Learning Rate & 0.001 \\
Scheduler & ReduceLROnPlateau (patience=5) \\
Batch Size & 32 \\
Epochs & 50 \\
Early Stopping & 10 epochs sans amélioration \\
\hline
\end{tabular}
\end{center}

\subsection{Résultats du Modèle}

\begin{tcolorbox}[colback=successgreen!10, colframe=successgreen, title=\textbf{Métriques de Performance}]
\begin{center}
\begin{tabular}{|l|l|l|}
\hline
\textbf{Métrique} & \textbf{Valeur} & \textbf{Interprétation} \\
\hline
\textbf{MSE} & 0.0083 & Erreur quadratique très faible \\
\textbf{RMSE} & 0.0909 & Erreur moyenne ~9\% \\
\textbf{MAE} & 0.0545 & Erreur absolue ~5.5\% \\
\textbf{R²} & \textbf{0.7003} & 70\% variance expliquée \\
\textbf{Test Loss} & 0.0080 & Bonne généralisation \\
\hline
\end{tabular}
\end{center}
\end{tcolorbox}

\textbf{Interprétation:}
\begin{itemize}
    \item \textbf{R² = 0.70:} Le modèle explique 70\% de la variance des données, ce qui est un bon score pour des données environnementales complexes
    \item \textbf{RMSE < 0.1:} L'erreur de prédiction est inférieure à 10\%, acceptable pour un système d'alerte
    \item \textbf{Pas de surapprentissage:} Test loss proche de training loss
\end{itemize}

\subsection{Division des Données}
\begin{center}
\begin{tabular}{|l|l|l|}
\hline
\textbf{Dataset} & \textbf{Pourcentage} & \textbf{Taille} \\
\hline
Train & 70\% & ~1729 séquences \\
Validation & 15\% & ~370 séquences \\
Test & 15\% & ~370 séquences \\
\hline
\end{tabular}
\end{center}

% ==================== SECTION 5: FLUX DE DONNÉES ====================
\section{Flux de Données et Communication}

\subsection{Workflow Global}

\begin{enumerate}
    \item \textbf{Collecte:} Service Capteurs reçoit données MQTT des IoT
    \item \textbf{Traitement:} Service Satellite télécharge et analyse images Sentinel-2
    \item \textbf{Fusion:} STModel récupère données via API REST
    \item \textbf{Prédiction:} LSTM génère score qualité pour J+1
    \item \textbf{Publication:} Résultat publié sur Redis
    \item \textbf{Alertes:} Service Alertes consomme Redis et notifie si seuil dépassé
    \item \textbf{Visualisation:} API-SIG affiche sur carte interactive
\end{enumerate}

\subsection{Seuils OMS Surveillés}

\begin{center}
\begin{tabular}{|l|l|l|l|}
\hline
\textbf{Paramètre} & \textbf{Seuil OMS} & \textbf{Unité} & \textbf{Source} \\
\hline
pH & 6.5 - 8.5 & - & Capteurs \\
Turbidité & < 4.0 & NTU & Capteurs + Satellite \\
Température & < 25.0 & °C & Capteurs \\
Chlorophylle & Variable & mg/m³ & Satellite \\
NDWI & 0.0 - 1.0 & - & Satellite \\
\hline
\end{tabular}
\end{center}

\subsection{Communication Inter-Services}

\begin{itemize}
    \item \textbf{Synchrone:} API REST (HTTP/JSON)
    \item \textbf{Asynchrone:} Redis Pub/Sub
    \item \textbf{Temps réel:} WebSockets
    \item \textbf{Format:} JSON, GeoJSON
\end{itemize}

% ==================== SECTION 6: API ENDPOINTS ====================
\section{API Endpoints Principaux}

\subsection{Service Capteurs (8001)}
\begin{lstlisting}
GET  /health                    # Health check
GET  /api/capteurs              # Liste capteurs
GET  /api/capteurs/positions    # GPS capteurs (pour carte)
GET  /api/capteurs/data/latest  # Dernieres mesures (pour ML)
POST /api/capteurs/mesures      # Ajouter mesure
\end{lstlisting}

\subsection{Service Satellite (8002)}
\begin{lstlisting}
GET  /health                         # Health check
GET  /api/satellite/indices/latest   # Derniers indices
GET  /api/satellite/images           # Liste images
\end{lstlisting}

\subsection{Service STModel (8003)}
\begin{lstlisting}
GET  /health                    # Health check
GET  /api/model/info            # Info modele (architecture, metriques)
POST /api/predictions/create    # Creer prediction (14 mesures)
GET  /api/predictions/latest    # Dernieres predictions
\end{lstlisting}

\subsection{Service Alertes (8004)}
\begin{lstlisting}
GET  /health                    # Health check
GET  /api/alerts/history        # Historique alertes
POST /api/alerts                # Creer alerte manuelle
\end{lstlisting}

\subsection{Service API-SIG (8005)}
\begin{lstlisting}
GET  /health                    # Health check
GET  /api/zones                 # Zones geographiques
GET  /api/zones/[id]/quality    # Qualite par zone
\end{lstlisting}

% ==================== SECTION 7: DÉPLOIEMENT ====================
\section{Déploiement et Exécution}

\subsection{Prérequis}
\begin{itemize}
    \item Docker Desktop (obligatoire)
    \item Git
    \item Node.js 20+
    \item Python 3.12+
\end{itemize}

\subsection{Commandes de Démarrage}
\begin{lstlisting}[language=bash]
# Cloner le projet
git clone https://github.com/Hamza-Aglagal/AquaWatch-MS.git
cd AquaWatch-MS

# Configuration
copy .env.template .env

# Lancer infrastructure
docker compose up db_capteurs db_satellite db_predictions 
    db_alerts db_geo redis_queue minio_storage -d

# Lancer tous les services
docker compose up -d

# Verifier logs
docker compose logs -f service_stmodel
\end{lstlisting}

\subsection{URLs d'Accès}
\begin{center}
\begin{tabular}{|l|l|}
\hline
\textbf{Service} & \textbf{URL} \\
\hline
Capteurs API & http://localhost:8001 \\
Satellite API & http://localhost:8002 \\
STModel API & http://localhost:8003 \\
Alertes API & http://localhost:8004 \\
Carte Interactive & http://localhost:8005 \\
MinIO Console & http://localhost:9001 \\
GeoServer & http://localhost:8080 \\
\hline
\end{tabular}
\end{center}

% ==================== SECTION 8: QUESTIONS/RÉPONSES ====================
\section{Questions Fréquentes et Réponses}

\subsection{Questions sur l'Architecture}

\begin{tcolorbox}[colback=blue!5, colframe=aquablue]
\textbf{Q1: Pourquoi avoir choisi une architecture microservices?}

\textbf{R:} L'architecture microservices offre:
\begin{itemize}[noitemsep]
    \item \textbf{Modularité:} Chaque service est indépendant et peut être développé/déployé séparément
    \item \textbf{Scalabilité:} Possibilité de scaler uniquement les services surchargés
    \item \textbf{Résilience:} Une panne d'un service n'affecte pas les autres
    \item \textbf{Polyglottisme:} Choix de la meilleure technologie par service (Node.js pour I/O, Python pour ML)
    \item \textbf{Travail en équipe:} 3 développeurs travaillent en parallèle
\end{itemize}
\end{tcolorbox}

\begin{tcolorbox}[colback=blue!5, colframe=aquablue]
\textbf{Q2: Comment les services communiquent-ils entre eux?}

\textbf{R:} Deux modes de communication:
\begin{itemize}[noitemsep]
    \item \textbf{Synchrone (API REST):} Pour requêtes directes (ex: STModel appelle Capteurs pour données)
    \item \textbf{Asynchrone (Redis Pub/Sub):} Pour événements (ex: STModel publie prédiction, Alertes consomme)
\end{itemize}
Avantage: Découplage temporel entre producteur et consommateur de données.
\end{tcolorbox}

\begin{tcolorbox}[colback=blue!5, colframe=aquablue]
\textbf{Q3: Pourquoi utiliser TimescaleDB plutôt que PostgreSQL classique?}

\textbf{R:} TimescaleDB est optimisé pour les séries temporelles:
\begin{itemize}[noitemsep]
    \item \textbf{Hypertables:} Partitionnement automatique par temps
    \item \textbf{Compression:} Réduction stockage jusqu'à 95\%
    \item \textbf{Agrégations rapides:} Optimisé pour requêtes temporelles
    \item \textbf{Rétention:} Politique de suppression automatique des anciennes données
\end{itemize}
Idéal pour les mesures IoT à haute fréquence.
\end{tcolorbox}

\subsection{Questions sur le Machine Learning}

\begin{tcolorbox}[colback=purple!5, colframe=purple!70]
\textbf{Q4: Pourquoi avoir choisi un modèle LSTM?}

\textbf{R:} Le LSTM (Long Short-Term Memory) est idéal car:
\begin{itemize}[noitemsep]
    \item \textbf{Mémoire longue:} Capture les dépendances temporelles sur 14 jours
    \item \textbf{Séquences:} Conçu pour données ordonnées dans le temps
    \item \textbf{Oubli sélectif:} Mécanisme de portes (gates) pour retenir l'important
    \item \textbf{Robustesse:} Résiste au problème du gradient vanishing
\end{itemize}
Alternative considérée: GRU (plus simple) ou Transformer (plus puissant mais plus lourd).
\end{tcolorbox}

\begin{tcolorbox}[colback=purple!5, colframe=purple!70]
\textbf{Q5: Comment interpréter le R² de 0.70?}

\textbf{R:} Le coefficient R² = 0.70 signifie:
\begin{itemize}[noitemsep]
    \item Le modèle explique \textbf{70\% de la variance} des données
    \item 30\% reste inexpliqué (bruit, facteurs non mesurés)
    \item Pour des données environnementales complexes, c'est un \textbf{bon score}
    \item Comparaison: R² > 0.6 est considéré acceptable, R² > 0.8 excellent
\end{itemize}
\end{tcolorbox}

\begin{tcolorbox}[colback=purple!5, colframe=purple!70]
\textbf{Q6: Pourquoi utiliser 14 jours d'historique?}

\textbf{R:} Choix basé sur:
\begin{itemize}[noitemsep]
    \item \textbf{Cycles naturels:} 14 jours captent variations hebdomadaires
    \item \textbf{Disponibilité satellites:} Revisite Sentinel-2 tous les 5 jours
    \item \textbf{Compromis:} Assez long pour contexte, assez court pour réactivité
    \item \textbf{Expérimentation:} Tests avec 7, 14, 21 jours - 14 optimal
\end{itemize}
\end{tcolorbox}

\begin{tcolorbox}[colback=purple!5, colframe=purple!70]
\textbf{Q7: Qu'est-ce que la normalisation MinMaxScaler et pourquoi l'utiliser?}

\textbf{R:} MinMaxScaler transforme les données entre 0 et 1:
$$X_{norm} = \frac{X - X_{min}}{X_{max} - X_{min}}$$
\begin{itemize}[noitemsep]
    \item \textbf{Convergence:} Les réseaux de neurones apprennent mieux avec valeurs normalisées
    \item \textbf{Égalité:} Empêche les features à grande échelle de dominer
    \item \textbf{Stabilité:} Évite explosion/vanishing des gradients
\end{itemize}
Important: Conserver les scalers pour dénormaliser les prédictions!
\end{tcolorbox}

\begin{tcolorbox}[colback=purple!5, colframe=purple!70]
\textbf{Q8: Quelle est la différence entre MSE, RMSE et MAE?}

\textbf{R:}
\begin{itemize}[noitemsep]
    \item \textbf{MSE} (Mean Squared Error): $\frac{1}{n}\sum(y - \hat{y})^2$ - Pénalise fortement grandes erreurs
    \item \textbf{RMSE} (Root MSE): $\sqrt{MSE}$ - Même unité que les données
    \item \textbf{MAE} (Mean Absolute Error): $\frac{1}{n}\sum|y - \hat{y}|$ - Robuste aux outliers
\end{itemize}
Dans notre cas: RMSE = 0.09 signifie erreur moyenne de 9\% sur score normalisé.
\end{tcolorbox}

\subsection{Questions sur les Données}

\begin{tcolorbox}[colback=cyan!5, colframe=cyan!70]
\textbf{Q9: Quels sont les indices satellites calculés et leur signification?}

\textbf{R:}
\begin{itemize}[noitemsep]
    \item \textbf{NDWI} (Normalized Difference Water Index):
    $$NDWI = \frac{Green - NIR}{Green + NIR}$$
    Valeur > 0 = présence d'eau, < 0 = terre/végétation
    
    \item \textbf{Chlorophylle:} Concentration d'algues (mg/m³), indicateur d'eutrophisation
    
    \item \textbf{Turbidité:} Clarté de l'eau, affectée par sédiments et pollution
\end{itemize}
\end{tcolorbox}

\begin{tcolorbox}[colback=cyan!5, colframe=cyan!70]
\textbf{Q10: Comment sont fusionnées les données capteurs et satellites?}

\textbf{R:} Processus en 4 étapes:
\begin{enumerate}[noitemsep]
    \item \textbf{Alignement temporel:} Interpolation pour dates manquantes
    \item \textbf{Alignement spatial:} Association par coordonnées GPS (rayon 5km)
    \item \textbf{Jointure:} Fusion sur date + station\_id
    \item \textbf{Qualité:} Calcul du days\_diff (écart temporel capteur-satellite)
\end{enumerate}
Notebook: \texttt{02\_fusion\_optimized.ipynb}
\end{tcolorbox}

\subsection{Questions sur les Alertes}

\begin{tcolorbox}[colback=red!5, colframe=alertred]
\textbf{Q11: Comment fonctionne le système d'alertes?}

\textbf{R:} Flux d'alerte:
\begin{enumerate}[noitemsep]
    \item STModel publie prédiction sur Redis (canal \texttt{predictions})
    \item Service Alertes écoute Redis en continu
    \item Comparaison avec seuils OMS configurés
    \item Si dépassement: création alerte en base
    \item Envoi notification (email/SMS) aux destinataires de la zone
    \item Mise à jour statut (pending → sent → acknowledged)
\end{enumerate}
\end{tcolorbox}

\begin{tcolorbox}[colback=red!5, colframe=alertred]
\textbf{Q12: Quels types d'alertes sont gérés?}

\textbf{R:} 4 types configurés:
\begin{center}
\begin{tabular}{|l|l|l|}
\hline
\textbf{Code} & \textbf{Sévérité} & \textbf{Déclencheur} \\
\hline
WATER\_QUALITY\_BAD & High & Score qualité < seuil \\
PH\_CRITICAL & Critical & pH < 6 ou pH > 9 \\
POLLUTION\_RISK & Medium & Prédiction pollution \\
SENSOR\_OFFLINE & Low & Capteur sans données \\
\hline
\end{tabular}
\end{center}
\end{tcolorbox}

\subsection{Questions Techniques Docker}

\begin{tcolorbox}[colback=gray!10, colframe=gray!50]
\textbf{Q13: Comment fonctionne Docker Compose dans ce projet?}

\textbf{R:} Docker Compose orchestre 12+ conteneurs:
\begin{itemize}[noitemsep]
    \item \textbf{depends\_on:} Définit ordre de démarrage (DB avant services)
    \item \textbf{networks:} Réseau interne pour communication
    \item \textbf{volumes:} Persistance données (bases, MinIO)
    \item \textbf{environment:} Variables de configuration
    \item \textbf{healthcheck:} Vérification état des services
\end{itemize}
Commande: \texttt{docker compose up -d} lance tout.
\end{tcolorbox}

\begin{tcolorbox}[colback=gray!10, colframe=gray!50]
\textbf{Q14: Comment déboguer un service qui ne démarre pas?}

\textbf{R:} Étapes de diagnostic:
\begin{lstlisting}[language=bash]
# 1. Voir logs du service
docker compose logs -f service_stmodel

# 2. Verifier containers actifs
docker compose ps

# 3. Entrer dans container
docker compose exec service_stmodel bash

# 4. Verifier connexion DB
docker compose exec db_predictions psql -U user -d db
\end{lstlisting}
\end{tcolorbox}

\subsection{Questions sur l'Évolution}

\begin{tcolorbox}[colback=green!5, colframe=successgreen]
\textbf{Q15: Quelles améliorations futures sont prévues?}

\textbf{R:} Roadmap:
\begin{enumerate}[noitemsep]
    \item \textbf{Phase 2:} Authentification (JWT, rôles admin/autorité/citoyen)
    \item \textbf{Phase 3:} Interface citoyens (vue simplifiée, signalements)
    \item \textbf{Phase 4:} Dashboard avancé (rapports PDF, export Excel)
    \item \textbf{Phase 5:} PWA mode hors-ligne, notifications push
    \item \textbf{ML:} Test modèles Transformer, prédictions 72h
\end{enumerate}
\end{tcolorbox}

\begin{tcolorbox}[colback=green!5, colframe=successgreen]
\textbf{Q16: Comment améliorer le R² du modèle?}

\textbf{R:} Pistes d'amélioration:
\begin{itemize}[noitemsep]
    \item \textbf{Plus de données:} Augmenter historique d'entraînement
    \item \textbf{Features:} Ajouter météo, débit, saison
    \item \textbf{Architecture:} Tester Transformer, Attention mechanism
    \item \textbf{Ensemble:} Combiner plusieurs modèles (bagging)
    \item \textbf{Hyperparamètres:} Optimisation bayésienne
\end{itemize}
\end{tcolorbox}

% ==================== SECTION 9: GLOSSAIRE ====================
\section{Glossaire}

\begin{description}[style=nextline]
    \item[API REST] Interface de programmation utilisant HTTP (GET, POST, PUT, DELETE)
    \item[CORS] Cross-Origin Resource Sharing - Politique de sécurité navigateur
    \item[FastAPI] Framework Python moderne pour APIs (async, validation Pydantic)
    \item[GeoJSON] Format JSON pour données géographiques
    \item[Hypertable] Table TimescaleDB partitionnée automatiquement par temps
    \item[IoT] Internet of Things - Objets connectés (capteurs)
    \item[LSTM] Long Short-Term Memory - Réseau de neurones récurrent
    \item[MQTT] Protocole léger de messagerie pour IoT
    \item[NDWI] Normalized Difference Water Index
    \item[OMS] Organisation Mondiale de la Santé
    \item[PostGIS] Extension PostgreSQL pour données géospatiales
    \item[Redis] Base de données in-memory, cache et pub/sub
    \item[Sentinel-2] Satellites d'observation terrestre européens
    \item[TimescaleDB] Extension PostgreSQL pour séries temporelles
\end{description}

% ==================== CONCLUSION ====================
\section{Conclusion}

AquaWatch-MS est une plateforme complète de surveillance de la qualité de l'eau combinant:
\begin{itemize}
    \item \textbf{IoT:} Collecte temps réel via capteurs
    \item \textbf{Satellite:} Analyse images Sentinel-2
    \item \textbf{Machine Learning:} Prédiction LSTM (R² = 0.70)
    \item \textbf{Alertes:} Notifications automatiques OMS
    \item \textbf{Cartographie:} Visualisation interactive
\end{itemize}

L'architecture microservices permet une évolution modulaire et un travail d'équipe efficace. Le projet démontre l'intégration réussie de technologies modernes (Docker, FastAPI, PyTorch, PostGIS) pour résoudre un problème environnemental concret.

\vspace{1cm}
\begin{center}
\textit{Document préparé pour révision - Décembre 2025}
\end{center}

\end{document}
